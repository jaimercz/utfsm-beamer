\documentclass[
	9pt,
	% handout % Uncomment to produce Handouts
]{beamer}

%%% CONFIGURATION
\newcommand{\TheLogo}{figures/logoutfsmind_v.png}	%Required
\newcommand{\TheBackground}{backgrounds/bgWhite.png}


%%% HANDOUTS
% Uncomment the following lines to prepare '3 slides in 1' Handouts
% It is also advisable to add the option [handout] to the document class.
% Options: `2 slides in 1`, `3 slides in 1` `4 slides in 1`
%
\usepackage{etex}
%\usepackage{handoutWithNotes}
%\pgfpagesuselayout{3 on 1 with notes}[letterpaper,border shrink=5mm]

\usepackage[spanish]{babel}
\usepackage[utf8,utf8x]{inputenc}

\deftranslation[to=spanish]{Example}{Ejemplo}
\deftranslation[to=spanish]{Definition}{Definición}
\deftranslation[to=spanish]{Theorem}{Teorema}


%%% Transitions
\mode<presentation>
{
	\usetheme{JCR}
	\setbeamercovered{transparent = 28}
	\beamerdefaultoverlayspecification{<+->}
	% Global Background
	\usebackgroundtemplate{%
    	\includegraphics[width=\paperwidth,height=\paperheight]{\TheBackground}%
	}
}

%\usefonttheme{serif}
%\usecolortheme{default}
%\usecolortheme{orchid}
\setbeamercolor{block title}{bg=gray40!10,fg=red!70}
\setbeamercolor{block body}{bg=gray!5}
\setbeamercolor{block title alerted}{bg = black,fg=red}
\setbeamercolor{block body alerted}{bg = black!40,fg = white!80}


\usepackage{amssymb,amsmath}
\usepackage{epstopdf}
%\usepackage{epsfig}


\DeclareMathOperator*{\maxim}{Max}
\DeclareMathOperator*{\argmax}{arg\,max}
\numberwithin{equation}{section}


\title{Título de la Presentación}
\subtitle{Subtítulo de la Presentación}
\author{Jaime C. Rubin-de-Celis}
\institute{Universidad Técnica Federico Santa María}
\date{\today}
%\titlegraphic{\includegraphics[width=40mm]{\TheLogo}}


%---------------------------------------------------------------------------
%%% PDF OPTIONS (HYPERREFERENCES)
\hypersetup{
		colorlinks=false
		,linkcolor=black!80
		,urlcolor=blue
		,citecolor=red
		,breaklinks=true
  		,pdfinfo={  
    Title={Apuntes de Gestión Estratégica},  
    Subject={Gestión Estratégica},
    Author={Jaime C. Rubin-de-Celis}
    Keywords={Beamer Template},
		Producer={JCR's LaTeX, http://www.rubin-de-celis.com/},
		pdfpagemode=FullScreen,
		pdfmenubar=false,
		pdftoolbar=false
  }  
}
%---------------------------------------------------------------------------


%---------------------------------------------------------------------------
%%% FIGURES, GRAPHICS AND COLORS
\usepackage{wrapfig}			% flooating Figures and Tables
	% Colors
	\definecolor{gray80}{gray}{0.80}
	\definecolor{gray40}{gray}{0.40}
%---------------------------------------------------------------------------

%---------------------------------------------------------------------------
%%% TIKZ
\usepackage{tikz}				% to hard to explain ;-)
\usetikzlibrary{arrows,positioning,fit,shapes,calc,backgrounds}
\usetikzlibrary{decorations.pathreplacing}
\tikzstyle{line} = [draw, thick, gray40, ->, >=latex]
% ATTENTION: library conflicts with pgfpages!!!
%---------------------------------------------------------------------------


%---------------------------------------------------------------------------
%%% Begin Document
%---------------------------------------------------------------------------

\begin{document}

\frame{\titlepage}

%%%
\section{Introducción}

%
\begin{frame}
\frametitle{Introducción}

Introducción ...

\begin{itemize}
\item Itemize
\item Itemize
\item Itemize
\end{itemize}

\begin{alertblock}{Alerta}
\centering
Alert Block
\end{alertblock}

\end{frame}


%
\begin{frame}
\frametitle{Introducción}
\begin{enumerate}
\item Enumerate
\item Enumerate
\item Enumerate
\end{enumerate}

\begin{block}{Bloque}
\centering
Block Generico
\end{block}

\end{frame}


%%%
\section{Sección I}

%
\begin{frame}
\frametitle{Sección I}
\begin{columns}[t]
	\column{.5\textwidth}
	\begin{block}{Ventajas}
	\begin{itemize}
	\item Columnas
	\end{itemize}
	\end{block}
	\column{.5\textwidth}
	\begin{block}{Desventajas}
	\begin{itemize}
	\item Columnas
	\end{itemize}
	\end{block}
\end{columns}
\end{frame}

%
\begin{frame}
\frametitle{Sección I}
\framesubtitle{Subtítulo Sección I}
	Bloques de Ejemplo\footnote{Esta es una nota al pie.}.
	\begin{example}
	Ejemplo\footnote{Esta es una nota al pie.} ...
	\begin{equation*}
	E =mc^2
	\end{equation*}

	\end{example}
\end{frame}

%
\begin{frame}
\frametitle{Sección I}
\framesubtitle{Subtítulo Sección I}
	\begin{example}
	Ejemplo figuras ...
	
	\begin{figure}
	\centering
	\includegraphics[width=.85\textwidth]{figures/logoutfsmind_h.png}
	\caption{Incorporación de Figuras {\tiny (Fuente: Elaboración Propia.)}}
	\end{figure}
	
	\end{example}

\end{frame}

\end{document}
